\documentclass{article}
\usepackage[margin=0.5in]{geometry}
\usepackage[utf8]{inputenc}   % поддержка UTF8
\usepackage[T2A]{fontenc}     % внутренняя T2A кодировка TeX
\usepackage[english, russian]{babel}
\usepackage{amsmath}

\newcommand{\ldsim}{\texttt{ldsim}}
\newcommand{\Ber}{\mathcal{B}}

\begin{document}

\section{Дрейф-диффузионная модель транспорта} 

\subsection{Общее описание}
Основная часть модели лазерного диода в \ldsim{} -- это дрейф-диффузионная
модель транспорта свободных носителей заряда в полупроводнике. Эта модель
состоит из уравнений непрерывности плотностей тока электронов $\mathbf{j_n}$
и дырок $\mathbf{j_p}$, а также уравнения Пуассона:
\begin{equation*}
	-\nabla{\left( \varepsilon\varepsilon_0  \nabla \psi \right)} =
    q \left( C - n + p \right).
\end{equation*}
\begin{equation*}
	\nabla{}\mathbf{j_n} - q\frac{\partial{n}}{\partial{t}} = qR,
\end{equation*}
\begin{equation*}
	\nabla{}\mathbf{j_p} + q\frac{\partial{p}}{\partial{t}} = -qR,
\end{equation*}
Здесь $n$, $p$ -- концентрации свободных электронов и дырок, соответственно;
$\psi$ -- электростатический потенциал;
$C = N_D^+ - N_A^-$ -- концентрация ионизированных примесей;
$R$ -- скорость рекомбинации; 
$q$ -- элементарный заряд;
$\varepsilon\varepsilon_0$ -- диэлектрическая проницаемость.

Мы рассматриваем только стационарную задачу, поэтому производные концентраций
по времени ($\frac{\partial n}{\partial t}$ и $\frac{\partial p}{\partial t}$)
везде равны $0$. Также в текущей версии \ldsim{} моделирование транспорта
осуществляется только в вертикальном направлении $x$, т.е. в направлении роста
эпитаксиальной структуры. В результате получается одномерная стационарная модель,
описываемая следующими уравнениями:
\begin{equation}
    \label{eq:poisson}
    -\frac{d}{dx}
    {\left( \varepsilon\varepsilon_0  \frac{d\psi}{dx} \right)} =
    q \left( C - n + p \right).
\end{equation}
\begin{equation}
    \label{eq:jn_cont}
    \frac{dj_n}{dx} = qR,
\end{equation}
\begin{equation}
    \frac{dj_p}{dx} = -qR,
    \label{eq:jp_cont}
\end{equation}

\subsection{Вид решения и граничные условия}
Расчёт концентраций свободных носителей заряда осуществляется по следующим
формулам:
\begin{equation*}
    n = N_c F_{1/2}(\eta_n), \qquad
    \eta_n = \frac{E_{fn} - E_c}{kT} = \frac{q(\psi - \varphi_n) - E_c}{kT};
\end{equation*}
\begin{equation*}
    p = N_v F_{1/2}(\eta_p), \qquad
    \eta_p = \frac{E_v - E_{fp}}{kT} = \frac{E_v - q(\psi - \varphi_p)}{kT}.
\end{equation*}
Здесь
$N_{c,v} = 2 \left( \frac{2\pi m_{n,p}^{*} kT}{h^2} \right)^{3/2}$ --
эффективные плотности состояний в зоне проводимости и валентной зоне;
$F_{1/2}(\eta) = \frac{2}{\sqrt{\pi}}\int_0^\infty{\frac{x^{1/2}dx}{1+\exp(x-\eta)}}$
-- интеграл Ферми степени $1/2$;
$E_{fn,fp}$ -- квазиуровни Ферми для электронов и дырок;
$E_c$ -- дно зоны проводимости;
$E_v$ -- потолок валентной зоны;
$\varphi_{n,p}$ -- квазипотенциалы Ферми для электронов и дырок;
$k$ -- постоянная Больцмана; $T$ -- температура.

В результате решением системы (\ref{eq:poisson}--\ref{eq:jp_cont}) является
тройка значений $(\psi, \varphi_n, \varphi_p)$ для всех точек структуры.
Граничные условия для решения такого вида запишутся как:
\begin{align*}
    \psi(0) = \psi_{bi}(0) + V_1, \qquad& \varphi_n(0) = \varphi_p(0) = V_1,\\
    \psi(d) = \psi_{bi}(d) + V_2, \qquad& \varphi_n(d) = \varphi_p(d) = V_2.
\end{align*}
Здесь $d$ -- толщина эпитаксиальной структуры;
$V_{1,2}$ -- напряжения (потенциалы) прикладываемые в $x = 0$ и $x = d$,
т.е. внешнее напряжение $V_{ext} = V_2 - V_2$;
$\psi_{bi}$ -- встроенный потенциал, получаемый при решении уравнения
Пуассона \eqref{eq:poisson} для $V_{ext} = 0$.
Из выбранного вида решения следуют граничные условия на гетеропереходах --
непрерывность квазиуровней Ферми $E_{fn}$ и $E_{fp}$.

\subsection{Спонтанная рекомбинация}
При расчёте скорости рекомбинации $R$ учитываются 3 механизма:
\begin{equation*}
    R = R_{SRH} + R_{rad} + R_{Aug},
\end{equation*}
рекомбинация Шокли--Рида--Холла ($R_{SRH}$), излучательная рекомбинация
($R_{rad}$) и Оже-рекомбинация ($R_{Aug}$). Для их расчёта используются
следующие формулы:
\begin{equation*}
    R_{SRH} = \frac{np - n_1 p_1}{(n + n_1)\tau_p + (p + n_1)\tau_n},
    \quad
    n_1 = n_0 \frac{1-f_{t0}}{f_{t0}},
    \quad
    p_1 = p_0 \frac{f_{t0}}{1-f_{t0}},
    \quad
    f_{t0} = \frac{1}{1 + \exp{\left(\frac{E_t-E_f}{kT}\right)}};
\end{equation*}
\begin{equation*}
    R_{rad} = B(np - n_0 p_0);
\end{equation*}
\begin{equation*}
    R_{Aug} = (C_n n + C_p p)(np - n_0 p_0).
\end{equation*}
Здесь $n_0$, $p_0$ -- равновесные концентрации;
$\tau_{n,p}$ -- временя захвата электронов и дырок на глубокие уровни
(ловушики захвата);
$f_{t0}$ -- доля заполненных ловушек захвата, расположенных на уровне $E_t$;
$B$ -- коэффициент излучательной рекомбинации;
$C_{n,p}$ -- коэффициенты Оже-рекомбинации.

\subsection{Плотность тока и схема дискретизации}
Приведём по две формулы для плотностей тока электронов $j_n$ и дырок $j_p$,
первая записывается как сумма дрейфовой и диффузионной компонент, вторая
определяет плотность тока через изменение квазипотенциала.
\begin{equation*}
	j_n = -q \mu_n n \frac{d\psi}{dx} + q D_n \frac{dn}{dx} =
	      -q \mu_n n \frac{d\varphi_n}{dx};
    \qquad
	j_p = -q \mu_p p \frac{d\psi}{dx} - q D_p \frac{dp}{dx} =
	      -q \mu_p p \frac{d\varphi_p}{dx}.
\end{equation*}

В уравнения (\ref{eq:jn_cont}, \ref{eq:jp_cont}) входят производные плотностей
тока по координате $x$. Из-за того, что стандартная дискретизация с помощью
конечных разностей приводит к нестабильности системы, обычно при моделировании
дрейф-диффузионного транспорта используют схему Шарфеттера--Гуммеля:
\begin{equation*}
    j_{n;i,i+1} = -\frac{q \mu_n V_T}{h_{i,i+1}}
                   \left[ \Ber\left(-\frac{\psi_{i+1}-\psi_i}{V_T}\right) n_i
                         -\Ber\left( \frac{\psi_{i+1}-\psi_i}{V_T}\right) n_{i+1}
                   \right];
\end{equation*}
\begin{equation*}
j_{p;i,i+1} = \frac{q \mu_p V_T}{h_{i,i+1}}
			  \left[ \Ber\left(  \frac{\psi_{i+1}-\psi_i}{V_T} \right) p_i
                    -\Ber\left( -\frac{\psi_{i+1}-\psi_i}{V_T} \right) p_{i+1}
              \right].
\end{equation*}
Здесь $\Ber(x) = x / (\exp{x}-1)$ -- функция Бернулли,
$V_T = kT/q$ -- тепловой потенциал.
Таким образом, используется метод конечных объёмов -- значения плотностей тока
определяются только на границах объёмов через потенциалы внутри объёмов.

Приведённые выше формулы Шарфеттера--Гуммеля были получены из предположения,
что свободные носители подчиняются статистике Больцмана. Это упрощение
справедливо только в случае, когда квазиуровни Ферми лежат в запрещённой
зоне, а расстояние до границы зоны $\gg kT$. Очевидно, что это условие
будет соблюдаться далеко не всегда, в частности, при сильном вырождении,
наблюдаемом в активных областях полупроводниковых лазеров, получаемые
по формуле Шарфеттера--Гуммеля значения плотностей тока будут сильно завышеными.
Аналитической формулы для статистики Ферми--Дирака не существует, поэтому
в расчётах часто используются модифицированные схемы Шарфеттера--Гуммеля.
В \ldsim{} она имеет следущий вид:
\begin{equation*}
j_{n;i,i+1} = -\frac{q \mu_n V_T}{h_{i,i+1}}
               N_c \sqrt{ \frac {F_{1/2}(\eta_{n;i}) F_{1/2}(\eta_{n;i+1})}
                                {\exp{(\eta_{n;i})} \exp{(\eta_{n;i+1})}} }
               \left[ \Ber\left(-\frac{\psi_{i+1}-\psi_i}{V_T}\right) \exp{(\eta_{n;i})}
                     -\Ber\left( \frac{\psi_{i+1}-\psi_i}{V_T}\right) \exp{(\eta_{n;i+1})}
               \right].
\end{equation*}
Аналогичная формула используется для расчёта плотности тока дырок.

В результате система (\ref{eq:poisson}--\ref{eq:jp_cont}) запишется в следующем виде:
\begin{equation}
\varepsilon_i \left( \frac{1}{h_{i,i+1}} \psi_{i+1} -
                   \left[\frac{1}{h_{i,i+1}}+\frac{1}{h_{i-1,i}}\right] \psi_{i} +
                   \frac{1}{h_{i-1,i}} \psi_{i-1} \right)
+\frac{q}{\varepsilon_0} \left( C_i + p(\psi_i, \varphi_{p;i})
                                - n(\psi_i, \varphi_{n;i})
                         \right) \omega_i
= 0;
\end{equation}
\begin{equation}
qR \omega_i - (j_{n;i,i+1} - j_{n;i-1,i}) = 0;
\end{equation}
\begin{equation}
-qR \omega_i - (j_{p;i,i+1} - j_{p;i-1,i}) = 0.
\end{equation}
Здесь $h_{i,i+1} = x_{i+1} - x_i$ -- расстояние между узлами сетки,
$\omega_i = x_{i;i+1} - x_{i-1;i}$ -- 1-мерный объём.

\section{Лазерная генерация}

В этом разделе описаны модификации, вносимые в описанную выше дрейф-диффузионную
систему для моделирования полупроводникового лазера. Рассматриваются
две модели: 1-мерная (1D) и 2-мерная (2D). В 1D случае уравнение, описывающее
лазерную генерацию, нульмерное, поэтому решение описывается единственным
числом -- концентрацией фотонов в волноводе $S$ -- а также зависимостями
$\psi(x)$, $\varphi_n(x)$ и $\varphi_p(x)$. Именно из-за наличия зависимости
части решения от $x$ эта модель и называется 1-мерной. В 2-мерном случае
все перечисленные параметры являются функциями от продольной (вдоль оси
резонатора) координаты $z$. При этом соседние точки по $z$-координате
связаны только через концентрацию фотонов, продольный транспорт не учитывается.
Такое упрощение не является критичным, при используемой формулировке задачи
в большинстве случаев транспорт относительно $z$ действительно будет
пренебрежимо мал.

\subsection{1D модель}
В данном случае лазерная генерация описывается с помощью скоростного уравнения
для концентрации фотонов в резонаторе $S$:
\begin{equation}
    \frac{\partial{S}}{\partial{t}} =
    v_g (\Gamma g - \alpha_{int} - \alpha_m) S + \beta_{sp} R_{rad}.
    \label{eq:photons_0D}
\end{equation}
Здесь $v_g$ -- групповая скорость фотонов в резонаторе;
$\Gamma$ -- коэффициент оптического ограничения, т.е. доля лазерной моды,
располагающаяся в активной области лазера;
$g$ -- материальное усиление;
$\alpha_{int}$ -- внутренние оптические потери;
$\alpha_m = \frac{1}{2L} \ln{\frac{1}{R_1R_2}}$ -- потери на вывод излучения,
$L$ -- длина резонатора, $R_{1,2}$ -- коэффициенты отражения зеркал;
$\beta_{sp}$ -- коэффициент спонтанного излучения, т.е. доля спонтанного
излучения, входящего в лазерную моду.

Материальное усиление $g$ зависит от концентраций электронов и дырок. В \ldsim{}
для его расчёта используется следующая модификация стандартной логарифмической
формулы
\begin{equation*}
    g(n, p) = g_0 \ln{\frac{\min{(n, p)}}{n_{tr}}},
\end{equation*}
где коэффициент усиления $g_0$ и концентрация прозрачности $n_{tr}$ являются
характеристиками активной области.

\ldots

\subsection{2D модель}
\ldots

\end{document}